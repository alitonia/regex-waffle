\documentclass{article}
\usepackage[utf8]{inputenc}
\usepackage[serbian]{babel} 
\usepackage{listings}
\usepackage{graphicx}
\usepackage{hyperref}
\hypersetup{
colorlinks,
linkcolor=blue,
urlcolor=blue
}
\setlength{\textheight}{600pt}
\setlength{\textwidth}{140mm}
\setlength{\topmargin}{5pt}
\setlength{\evensidemargin}{53pt}
\setlength{\oddsidemargin}{10mm}

\title{%
  Playing Regex Golf with Genetic Programming \vspace{0.4cm} \\ 
  \large Projekat u okviru kursa Računarska inteligencija \\
  Matematički fakultet\\ Univerzitet u Beogradu \vspace*{0.5cm}}
  
\author{Anđela Ilić \\
\href{mailto:mi17105@alas.matf.bg.ac.rs}{mi17105@alas.matf.bg.ac.rs} \\
Mina Milošević \\
\href{mailto:mi17081@alas.matf.bg.ac.rs}{mi17081@alas.matf.bg.ac.rs} \\
}

\date{\vspace*{1cm}Februar 2021}

\begin{document}

\maketitle

\newpage

\renewcommand*\contentsname{Sadržaj}
\tableofcontents
\newpage

\section{Opis problema}
Data su dva skupa reči - M i U. Cilj \textit{Regex Golf} igre je pronaći najkraći regularni izraz kojim se mogu zapisati sve reči iz skupa M, ali kojim se ne može zapisati nijedna reč skupa U. Za date skupove M i U ne možemo sa sigurnošću da tvrdimo da postoji rešenje koje zadovoljava prethodne uslove. Takođe, ako dobijemo regularni izraz koji zadovoljava navedene uslove, ne možemo za svaki primer znati da li postoji i bolje rešenje tj. kraći regularni izraz. 

\section{Implementacija}
Svaka jedinka u Genetskom programiranju će biti predstavljena kao drvo.
U listovima nalaze elementi koje ćemo jednim imenom zvati \textit{Terminali} (terminal set), a u unutrašnjim čvorovima su elementi koje nazivamo \textit{Funkcije} (function set). \\
Skup funkcija sadrži operatore koji se mogu javiti u regularnim izrazima. Primeri takvih operatora su: $.*$, $.+$, $.?$, $.\{.,.\}+$, $(.)$, $[.]$,
$[\ \textasciicircum .]$, $..$, $.|.$. Tačka $.$ je mesto na kome se nalaze
deca u drvetu. \\
Skup terminala čine elementi koji zavise i koji ne zavise od ulaznih skupova M i U. Elementi koji su nezavisni - opsezi malih i velikih slova, opsezi brojeva u
regularnim izrazima, karakteri $\textasciicircum$ i $\$$, wildcard karakter
'$\%$' (kasnije se transformiše u .). Elementi skupa terminala koji su zavisni - skup karaktera iz M, parcijalni opsezi karaktera iz M i n-grami.

\subsection{Obrada ulaznih podataka}
Bez obzira na veličinu ulaznih skupova M i U i njihov sadržaj, implementacija prethodno navedenih zavisnih terminala je ista. \\
\textit{Skup karaktera iz M} sadrži sve karaktere koji se mogu naći u rečima iz M, bez ponavljanja i sortirani po engleskom alfabetu. \\
\textit{Parcijalni opsezi} se prave na osnovu skupa karaktera iz M. Potrebno je naći maksimalne podskupove tog skupa karaktera tako da se svi karakteri iz intervala $[c_f, c_l]$ nalaze u skupu karaktera iz M. $c_f$ je prvi karakter, a $c_l$ je poslednji karakter iz podskupa. Kao rezultat se vraćaju parcijalni opsezi u formatu $c_f - c_l$. \\
\textit{n-grami}. Pravimo skup svih n-grama dužine $2 \leq n \leq 4$ koji se mogu naći u M ili u U (ili oba). Svakom dobijenom n-gramu se dodelju vrednost koja predstavlja njegov \textit{score}. Za svaku reč iz M koja sadrži dati n-gram, njegov \textit{score} se uvećava za 1, a za svaku reč iz U koja ga sadrži, \textit{score} se umanjuje za 1. Nakon formiranja svih n-grama i određivanja njihovih vrednosti, sortiraju se opadajuće po vrednosti.
Potrebno je uzeti najmanji podskup n-grama tako da njihova ukupna vrednost (\textit{score}) bude jednaka bar dužini skupa M ($|M|$).

\subsection{Genetsko programiranje}
\subsubsection{Implementacija jedinke}
Svaka jedinka se predstavlja preko \textit{apstraktnog sintaksnog stabla} (AST).
U korenu stabla se nalazi karakter '$.$' i koren uvek ima dva deteta. Elementi u unutrašnjim čvorovima se biraju \textit{random} iz skupa \textit{Function}, a elementi u listovima su \textit{random} izabrani iz skupa \textit{Terminal}. \\
Na osnovu izabranog elementa za unutrašnji čvor dobijamo informaciju koliko dece će taj čvor imati - $.*$, $.+$, $.?$, $(.)$, $[.]$, $[\ \textasciicircum .]$ će imati jedno dete; $..$, $.|.$ će imati dva deteta; $.\{.,.\}+$ ima tri deteta. 
Od ovako kreiranog drveta se dobija niska koja predstavlja validan regularni izraz. \\
Klasa \textit{Individual} koja predstavlja jedinku takođe sadrži i brojeve $n_m$ i $n_u$ - broj reči iz skupova M i U, redom, koje su opisane dobijenim regularnim izrazom. \\
Za svaku jedinku računamo i \textit{fitnes} funkciju po formuli:
$$f(x) = w_i * (n_m - n_u) - length(r)$$
gde je $w_i$ unapred zadata konstanta za dati primer, a $length(r)$ je dužina regularnog izraza. Pošto želimo što kraći regularni izraz koji obuhvata sve reči iz M i nijednu reč iz U, ovako definisanu funkciju \textit{fitnes} maksimizujemo.

\subsubsection{Parametri genetskog programiranja}

\subsubsection{Selekcija, ukrštanje, mutacija}
Za \textit{selekciju} koristimo turnirsku selekciju veličine 7. Jedinke za selekciju biramo random i uzimamo najbolju jedinku tj. onu koja ima najveći fitnes među odabranim.

\section{Rezultati}
\section{Zaključak}
\section{Izvori}


\end{document}