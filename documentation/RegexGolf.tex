\documentclass{article}
\usepackage[utf8]{inputenc}
\usepackage[serbian]{babel} 
\usepackage{listings}
\usepackage{graphicx}
\usepackage{tikz}
\usepackage{color, colortbl}
\definecolor{Gray}{gray}{0.9}
\usepackage[table,xcdraw]{xcolor}
\usepackage{hyperref}
\hypersetup{
colorlinks,
linkcolor=blue,
urlcolor=blue
}
\setlength{\textheight}{600pt}
\setlength{\textwidth}{140mm}
\setlength{\topmargin}{5pt}
\setlength{\evensidemargin}{53pt}
\setlength{\oddsidemargin}{10mm}

\title{%
  Implementacija Regex Golf igre pomoću Genetskog programiranja \vspace{0.4cm} \\ 
  \large Projekat u okviru kursa Računarska inteligencija \\
  Matematički fakultet\\ Univerzitet u Beogradu \vspace*{0.5cm}}
  
\author{Anđela Ilić \\
\href{mailto:mi17105@alas.matf.bg.ac.rs}{mi17105@alas.matf.bg.ac.rs} \\
Mina Milošević \\
\href{mailto:mi17081@alas.matf.bg.ac.rs}{mi17081@alas.matf.bg.ac.rs} \\
}

\date{\vspace*{1cm}Februar 2021}

\begin{document}

\maketitle

\newpage

\renewcommand*\contentsname{Sadržaj}
\tableofcontents
\newpage

\section{Opis problema}
Data su dva skupa reči - M i U, ne nužno istih dimenzija. Cilj \textit{Regex Golf} igre je pronaći najkraći regularni izraz kojim se mogu zapisati sve reči iz skupa M, ali kojim se ne može zapisati nijedna reč skupa U.\\
Za date skupove M i U ne možemo sa sigurnošću da tvrdimo da postoji rešenje koje zadovoljava prethodne uslove. Takođe, ako dobijemo regularni izraz koji zadovoljava navedene uslove, ne možemo za svaki primer znati da li postoji i bolje rešenje tj. kraći regularni izraz. \\
U nastavku je dat opis naše implementacije rešenja pomoću Genetskog programiranja \cite{Bartoli}.

\section{Implementacija}
Svaka jedinka u Genetskom programiranju će biti predstavljena kao drvo.
U listovima nalaze elementi koje ćemo jednim imenom zvati \textit{Terminali} (terminal set), a u unutrašnjim čvorovima su elementi koje nazivamo \textit{Funkcije} (function set). \\
Skup funkcija sadrži operatore koji se mogu javiti u regularnim izrazima. Primeri takvih operatora su: $.*$, $.+$, $.?$, $.\{.,.\}+$, $(.)$, $[.]$,
$[\ \textasciicircum .]$, $..$, $.|.$. Tačka $.$ je mesto na kome se nalaze
deca u drvetu. \\
Skup terminala čine elementi koji zavise i koji ne zavise od ulaznih skupova M i U. Elementi koji su nezavisni - opsezi malih i velikih slova (a-z, A-Z), opsezi brojeva u
regularnim izrazima (0-9), karakteri $\textasciicircum$ i $\$$, wildcard karakter
'$\%$' (kasnije se transformiše u .). Elementi skupa terminala koji su zavisni - skup svih karaktera iz M, parcijalni opsezi karaktera iz M i n-grami iz skupova U i M.

\subsection{Obrada ulaznih podataka}
Bez obzira na veličinu ulaznih skupova M i U i njihov sadržaj, implementacija prethodno navedenih zavisnih terminala je ista. \\
\textit{Skup karaktera iz M} sadrži sve karaktere koji se mogu naći u rečima iz M, bez ponavljanja i sortirani po engleskom alfabetu. \\
\textit{Parcijalni opsezi} se prave na osnovu skupa karaktera iz M. Potrebno je naći maksimalne podskupove tog skupa karaktera tako da se svi karakteri iz intervala $[c_f, c_l]$ nalaze u skupu karaktera iz M. $c_f$ je prvi karakter, a $c_l$ je poslednji karakter iz podskupa. Kao rezultat se vraćaju parcijalni opsezi u formatu $c_f - c_l$. \\
\textit{n-grami}. Pravimo skup svih n-grama dužine $2 \leq n \leq 4$ koji se mogu naći u M ili u U (ili oba). Svakom dobijenom n-gramu se dodelju vrednost koja predstavlja njegov \textit{score}. Za svaku reč iz M koja sadrži dati n-gram, njegov \textit{score} se uvećava za 1, a za svaku reč iz U koja ga sadrži, \textit{score} se umanjuje za 1. Nakon formiranja svih n-grama i određivanja njihovih \textit{score}-ova, sortiraju se opadajuće po \textit{score}-u.
Potrebno je uzeti najmanji podskup n-grama tako da njihova ukupna vrednost (\textit{score}) bude jednaka bar dužini skupa M ($|M|$).

\subsection{Genetsko programiranje}
\subsubsection{Implementacija jedinke}
Svaka jedinka se predstavlja preko \textit{stabla}.
U korenu stabla se nalazi karakter '$.$' i koren uvek ima dva deteta. Elementi u unutrašnjim čvorovima se biraju nasumično iz skupa \textit{Function}, a elementi u listovima su nasumično izabrani iz skupa \textit{Terminal}. \newpage
Na osnovu izabranog elementa za unutrašnji čvor dobijamo informaciju koliko dece će taj čvor imati - $.*$, $.+$, $.?$, $(.)$, $[.]$, $[\ \textasciicircum .]$ će imati jedno dete; $..$, $.|.$  dva deteta; $.\{.,.\}+$ ima tri deteta. \\
Od ovako kreiranog drveta se dobija niska koja predstavlja validan regularni izraz.
Za svaku jedinku računamo \textit{fitnes} funkciju po formuli:
$$f(x) = w_i * (n_m - n_u) - length(r)$$
gde je $w_i$ statistički određena konstanta (kod nas ima vrednost 10), $n_m$ i $n_u$ brojevi reči iz skupova M i U, redom, koje su opisane dobijenim regularnim izrazom, a $length(r)$ je dužina regularnog izraza. Pošto želimo što kraći regularni izraz koji obuhvata sve reči iz M i nijednu reč iz U, ovako definisanu funkciju \textit{fitnes} maksimizujemo.

\subsubsection{Selekcija, ukrštanje, mutacija, elitizam}
Za \textit{selekciju} koristimo turnirsku selekciju veličine 7. Jedinke
za selekciju biramo nasumično i uzimamo najbolju jedinku tj. onu koja ima najveći fitnes među odabranim. \\
Koristimo \textit{jednopoziciono ukrštanje}. Najpre obilazimo stabla roditelja BFS-om i numerišemo čvorove. Uzimamo broj iz intervala $[0, min(len(parent1), len(parent2))]$. Broj 0 odgovara korenu stabla, pa postavljamo \textit{child1} na \textit{parent2}, a \textit{child2} na \textit{parent1}. Inače, ako smo dobili broj veći od 0, tražimo podstabla u roditeljima koja treba da razmene mesta. Na kraju treba proveriti da li su regularni izrazi ovako dobijene dece validni. U slučaju da neko dete nije validno, vraćamo odgovarajućeg roditelja umesto njega. \\
\textit{Mutaciju} radimo sa verovatnoćom 0.1. Biramo indeks čvora koji želimo da promenimo. Razlikujemo dva slučaja - kada je čvor iz skupa terminala i kada je iz skupa funkcija. U prvom slučaju, samo izaberemo nasumično novi terminal i ažuriramo vrednost čvora.
Ukoliko je čvor bio iz skupa funkcija, biramo novu vrednost iz istog skupa ali sada treba obratiti pažnju da li treba menjati i njegovu decu. Ako smo izabrali element iste kardinalnosti (isti broj dece), onda samo promenimo vrednost u čvoru. Ako se kardinalnost smanjila, odgovarajuću decu brišemo, a ako se kardinalnost povećala biramo po jednu vrednost iz skupa terminala i postavljamo vrednost dece na nju. Na kraju opet treba proveriti da li je dobijena jedinka validna. Ako nije, vraćamo jedinku za koju smo i pokrenuli mutaciju. \\
U našoj implementaciji koristimo i \textit{elitizam} kako bismo sačuvali najbolje jedinke u populaciji. Elitizmom čuvamo $20\%$ najboljih jedinki.

\subsubsection{Parametri genetskog programiranja}
Parametri su postavljeni na osnovu podataka iz \cite{Bartoli}. Za većinu primera sa ovim parametrima se dobijao veoma dobar rezultat, dok je za neke bilo potrebno povećati broj generacija ili populacija. \\
U algoritmu imamo populaciju od 500 jedinki i pravimo 1000 generacija.
Kao što je ranije rečeno, turnirska selekcija je dimenzije 7, a verovatnoća mutacija je 0.1. Za elitizam uzimamo $20\%$ jedinki.
Kako bismo dobili što bolje rešenje algoritam pozivamo za 32 populacije.

\subsection{Izlaz programa}
Iz svake populacije čuvamo sve jedinke kod kojih važi:
$$n_m - n_u = |M|$$
$n_m$ i $n_u$ su brojevi reči iz skupova M i U, redom, koje su opisane regularnim izrazom, a $|M|$ je veličina skupa M. \\
Dakle, želimo da sačuvamo sve jedinke kod kojih su ispunjena prva 2 uslova ovog problema. Ako u trenutnoj populaciji ne postoji takva jedinka, čuvamo samo najbolju jedinku na osnovu celokupnog fitnesa. \\
Na kraju rada algoritma biramo najbolju jedinku, od svih sačuvanih, po njenom ukupnom fitnesu i nju proglasavamo za rešenjem datog problema.

\section{Rezultati}
U sledećoj tabeli su zabeleženi naši rezultati za odabrane primere. Primeri su uglavnom preuzeti sa \cite{Examples}. 
\begin{center}
\begin{tabular}{ccccc}
\hline
\rowcolor[HTML]{C0C0C0} 
\multicolumn{1}{|c|}{\cellcolor[HTML]{C0C0C0}\textbf{PRIMER}} & \multicolumn{1}{c|}{\cellcolor[HTML]{C0C0C0}\textbf{NAŠE REŠENJE}} & \multicolumn{1}{c|}{\cellcolor[HTML]{C0C0C0}\textbf{SCORE}} & \multicolumn{1}{c|}{\cellcolor[HTML]{C0C0C0}\textbf{NAJBOLJE REŠENJE}}      & \multicolumn{1}{c|}{\cellcolor[HTML]{C0C0C0}\textbf{SCORE}} \\ \hline
\rowcolor[HTML]{DAE8FC} 
\multicolumn{1}{|c|}{\cellcolor[HTML]{DAE8FC}Plain strings}   & \multicolumn{1}{c|}{\cellcolor[HTML]{DAE8FC}foo}                   & \multicolumn{1}{c|}{\cellcolor[HTML]{DAE8FC}207}            & \multicolumn{1}{c|}{\cellcolor[HTML]{DAE8FC}foo}                            & \multicolumn{1}{c|}{\cellcolor[HTML]{DAE8FC}207}            \\ \hline
\rowcolor[HTML]{ECF4FF} 
\multicolumn{1}{|c|}{\cellcolor[HTML]{ECF4FF}Anchors}         & \multicolumn{1}{c|}{\cellcolor[HTML]{ECF4FF}k\$}                   & \multicolumn{1}{c|}{\cellcolor[HTML]{ECF4FF}208}            & \multicolumn{1}{c|}{\cellcolor[HTML]{ECF4FF}k\$}                            & \multicolumn{1}{c|}{\cellcolor[HTML]{ECF4FF}208}            \\ \hline
\rowcolor[HTML]{DAE8FC} 
\multicolumn{1}{|c|}{\cellcolor[HTML]{DAE8FC}It never ends*}  & \multicolumn{1}{c|}{\cellcolor[HTML]{DAE8FC}u\$}                   & \multicolumn{1}{c|}{\cellcolor[HTML]{DAE8FC}28}             & \multicolumn{1}{c|}{\cellcolor[HTML]{DAE8FC}u\textbackslash{}b}             & \multicolumn{1}{c|}{\cellcolor[HTML]{DAE8FC}27}             \\ \hline
\rowcolor[HTML]{ECF4FF} 
\multicolumn{1}{|c|}{\cellcolor[HTML]{ECF4FF}Ranges}          & \multicolumn{1}{c|}{\cellcolor[HTML]{ECF4FF}ff$|$.b$|$d[a-f]}      & \multicolumn{1}{c|}{\cellcolor[HTML]{ECF4FF}178}            & \multicolumn{1}{c|}{\cellcolor[HTML]{ECF4FF}\textasciicircum{}{[}a-f{]}*\$} & \multicolumn{1}{c|}{\cellcolor[HTML]{ECF4FF}202}            \\ \hline
\rowcolor[HTML]{DAE8FC} 
\multicolumn{1}{|c|}{\cellcolor[HTML]{DAE8FC}Abba}  & \multicolumn{1}{c|}{\cellcolor[HTML]{DAE8FC}st$|$.+u$|$z}                   & \multicolumn{1}{c|}{\cellcolor[HTML]{DAE8FC}142}             & \multicolumn{1}{c|}{\cellcolor[HTML]{DAE8FC}\textasciicircum{}{(?!(.)+
\textbackslash{}1)$|$ef}}             & \multicolumn{1}{c|}{\cellcolor[HTML]{DAE8FC}196}             \\ 
\hline
\rowcolor[HTML]{ECF4FF} 
\multicolumn{1}{|c|}{\cellcolor[HTML]{ECF4FF}A man, a plan}          & \multicolumn{1}{c|}{\cellcolor[HTML]{ECF4FF}\textasciicircum{}{r}$|$x$|$gg$|$oo}      & \multicolumn{1}{c|}{\cellcolor[HTML]{ECF4FF}90}            & \multicolumn{1}{c|}{\cellcolor[HTML]{ECF4FF}\textasciicircum{}{(.)[\textasciicircum{}{p}].*\$}} & \multicolumn{1}{c|}{\cellcolor[HTML]{ECF4FF}177}            \\ 
\hline
\rowcolor[HTML]{DAE8FC} 
\multicolumn{1}{|c|}{\cellcolor[HTML]{DAE8FC}Presidents**}  & \multicolumn{1}{c|}{\cellcolor[HTML]{DAE8FC}Bu$|$am$|$Ta$|$Har$|$N+i$|$vel}                   & \multicolumn{1}{c|}{\cellcolor[HTML]{DAE8FC}120}             & \multicolumn{1}{c|}{\cellcolor[HTML]{DAE8FC}}             & \multicolumn{1}{c|}{\cellcolor[HTML]{DAE8FC}390}             \\ 
\hline
\rowcolor[HTML]{ECF4FF} 
\multicolumn{1}{|c|}{\cellcolor[HTML]{ECF4FF}Movies***}          & \multicolumn{1}{c|}{\cellcolor[HTML]{ECF4FF}m $|$B?R$|$[AB?]}      & \multicolumn{1}{c|}{\cellcolor[HTML]{ECF4FF}48}            & \multicolumn{1}{c|}{\cellcolor[HTML]{ECF4FF}m $|$ [tn]$|$b} & \multicolumn{1}{c|}{\cellcolor[HTML]{ECF4FF}40}            \\ 
\hline
\rowcolor[HTML]{DAE8FC} 
\multicolumn{1}{|c|}{\cellcolor[HTML]{DAE8FC}Regions****}  & \multicolumn{1}{c|}{\cellcolor[HTML]{DAE8FC}br$|$-$|$os$|$L$|$P$|$t?il}                   & \multicolumn{1}{c|}{\cellcolor[HTML]{DAE8FC}104}             & \multicolumn{1}{c|}{\cellcolor[HTML]{DAE8FC}}             & \multicolumn{1}{c|}{\cellcolor[HTML]{DAE8FC}200}             \\ 
\hline
\multicolumn{1}{l}{}  
\end{tabular}
\end{center}
Vreme izvršavanja programa je nekoliko desetina minuta (najčešće 20-30 minuta). \vspace{0.2cm} \\
$*$ U primeru na sajtu \cite{Examples} nije dozvoljeno korišćenje karaktera $\$$, pa su rešenja drugačija. \\
$**$ Primer je preuzet sa \cite{Presidents}. Nije pronađen podatak o najboljem rešenju pa je zapisan idealan \textit{score}. \\
$***$ Primer i najbolje rešenje su preuzeti sa \cite{Movies}. Njihovo rešenje je \textit{case insensitive}, dok je naše \textit{case sensitive}. \\
$****$ Primer je preuzet sa \cite{Regions}. Nije pronađen podatak o najboljem rešenju pa je zapisan idealan \textit{score}.

\newpage
\addcontentsline{toc}{section}{Literatura}
\begin{thebibliography}{9}
\bibitem{Bartoli} 
Alberto Bartoli, Eric Medvet, Andrea De Lorenzo, and Fabiano Tarlao - 
\textit{Playing Regex Golf with Genetic Programming} (2014)
\bibitem{Examples} 
\href{https://alf.nu/RegexGolf}{Regex Golf Examples}
\bibitem{Solutions} 
\href{https://gist.github.com/Davidebyzero/9221685}{Regex Golf Solutions}
\bibitem{Presidents} 
\href{https://pastebin.com/EvycCQTB}{USA Election winners and losers}
\bibitem{Movies} 
\href{http://zegnat.github.io/xkcd1313/}{Star Wars and Star Trek titles}
\bibitem{Regions} 
\href{https://codegolf.stackexchange.com/questions/17855/regex-golf-regions-of-italy-vs-states-of-usa}{Regions of Italy vs. States of USA}

\end{thebibliography}


\end{document}